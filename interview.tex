\documentclass[appendix.tex]{subfiles}

\begin{document}
\begin{interviewer}
Alright, so my main questions surround what factors you actually want to actually go into it.
Uhh, the current ones---I've just written up a quick version---the current one's that go into it are gender diversity, grade diversity, and having the teacher in the advisory.
Are there any other factors that you want to go into it?
\end{interviewer}

\begin{client}
Umm, right now there are no additional factors that I want to go into it.
Student Council was supposed to figure out at this morning's meeting are there any other parameters for next year that they want to go into it.
I'd say right now just try to be flexible, potentially try to consider student requests, though I'd say that's probably on the lower end of the priority scale.
You also want to have some function where you can exclude.
So sometimes there are kids that you just don't want in the same advisories.
Whether they have had something between them, or are siblings, or just spend too much time together, it is important to be able to keep some kids apart.
And I think having a system that also allows you to move kids around, but as you move kids it also gives you, if you move this kid then this kid needs to move here.
I don't know how capable that is, but the capacity to, as you move and change the equation, that there is a kind of self-correcting aspect where it says \enquote{I you move this kid, this kid needs to go here}.
And if it can in some way show me the options that fit into the parameters without causing a larger domino effect.
\end{client}

\begin{interviewer}
Okay, then, also with teachers, I know there is a lot that goes into the pairings.
So, as far as I know they are supposed to be Male-Female pairs, they are supposed to be a DP and an MYP teacher, and then I don't know how much of it is just chemistry between the two.
\end{interviewer}

\begin{client}
I try to keep teachers from having the same partners too many times.
I try to keep there being some diversity and new experiences for them as well.
\end{client}

\begin{interviewer}
Okay, well, what I wanted to ask was, do you think---I'm hoping that I can make it so you just input the teacher pairs.
Because I think that would be a lot easier than having to generate them.
\end{interviewer}

\begin{client}
Okay, that's fine, great!
\end{client}

\begin{interviewer}
Okay, so, next, I'd like to talk about how to input the data into the program.
Is there a central source of truth for student schedules and information?
\end{interviewer}

\begin{client}
Right now it's a spreadsheet.
It has colors, but it's just a spreadsheet.
\end{client}

\begin{interviewer}
Um, yeah, I just got access to an Excel sheet yesterday that I think is the one used currently.
It has a lot of the information that I need, but there are a couple things that are missing and I'm not sure how I'd get that data.
For one, it doesn't indicate the gender, or the sex, of the student.
\end{interviewer}

\begin{client}
Yeah, I just have that in my head.
\end{client}

\begin{interviewer}
Okay, and I also think it doesn't have the grade level.
And those are pretty much the main factors that I'm working with.
\end{interviewer}

\begin{client}
I believe that it is color coded according to grade level right now.
So there would be a band of background color for Senior, a band for Juniors, \&c.

As for gender... I don't think I can give you that.
I've learned a lot about student data and what we can and can't do with it over the past few weeks, and I'm pretty sure that's not something I can do.
\end{client}

\begin{interviewer}
That's fine, I don't actually need the data, I just need you to give me some sort of template, a way that my program can expect the data to be stored.
I don't have to actually have it.

Will that be fine?
\end{interviewer}

\begin{client}
Okay, yeah, that would be fine, we can do that.
\end{client}
\end{document}
