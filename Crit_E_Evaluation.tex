\documentclass[advisory-ia.tex]{subfiles}

\begin{document}
  \section{Evaluation}
  \label{sec:eval}
  By in large my product is successful at fulfilling what my client had laid out as requirements in our first meeting (Appendix A), though it falls short in a few key ways.
  The Backend development was a large portion of the process of creating my product, and it has reached a point where it can satisfy all design requirements except for one that my client mentioned in passing.
  My client had hoped that when moving a student from one advisory to another after generation the program would be able to display a few suggestions for whom to move to compensate for the student imbalance.
  While I originally agreed to implement this in the meeting, soon afterwards I realized that with the structure of my program this would be highly inefficient.
  Beyond this, I had no conception of how to possibly display this on the Frontend, so it has not yet been (and probably won't be) implemented.
  When I later informally met with my client to discuss this issue, he accepted this compromise.

  I've met almost all of my success criteria, except that my Frontend does not currently allow the user to forbid students from being with one another.
  The Backend API supports this behavior, but the Frontend needs to implement a dialog menu for such functionality, which would require significant restructuring of the list on unallocated people.
  Currently it only shows teachers, but in order to make modifications to students before generation I will need to provide the user with some way of viewing a list of students.

  My client was very pleased with the Frontend design and the functionality of the Backend when demonstrated.
  \lastpagelabel
\end{document}
